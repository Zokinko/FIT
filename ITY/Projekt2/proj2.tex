\documentclass[a4paper,11pt,twocolumn]{article}
\usepackage[left=1.5cm,text={18cm, 25cm},top=2.5cm]{geometry}
\usepackage[utf8]{inputenc}
\usepackage[IL2]{fontenc}
\usepackage[czech]{babel}
\usepackage{mathptmx}
\usepackage{amsthm}
\usepackage{amsmath}
\usepackage{hyperref}
\usepackage{times}
\usepackage{scalerel,amssymb}
\usepackage{hyperref} 		

\begin{document}
\begin{titlepage}
\begin{center}
\textsc{\Huge Fakulta informačních technologií\\
Vysoké učení technické v~Brně}\\
\vspace{\stretch{0.382}}
\LARGE Typografie a publikování -- 2. projekt\\
Sazba dokumentů a matematických výrazů\\
\vspace{\stretch{0.618}}
\Large 2019 \hfill         Tomáš Hrúz (xhruzt00)
\newpage
\end{center}
\end{titlepage}

\section*{Úvod}
V~této úloze si vyzkoušíme sazbu titulní strany, matematických vzorců, prostředí a dalších textových struktur obvyklých pro technicky zaměřené texty (například rovnice (1)
nebo Definice 1 na straně 1). Pro odkazovaní na vzorce
a struktury zásadně používáme příkaz \verb/\label/ a \verb/\ref/
případně\verb/ \pageref/ pokud se chceme odkázat na stranu
výskytu.

\par Na titulní straně je využito sázení nadpisu podle optického středu s~využitím zlatého řezu. Tento postup byl
probírán na přednášce. Dále je použito odřádkování se
zadanou relativní velikostí 0.4 em a 0.3 em.

\section{Matematický text}
Nejprve se podíváme na sázení matematických symbolů
a výrazů v~plynulém textu včetně sazby definic a vět s~využitím balíku \verb/amsthm/. Rovněž použijeme poznámku pod
čarou s~použitím příkazu \verb/\footnote/. Někdy je vhodné
použít konstrukci \verb/\mbox{}/, která říká, že text nemá být
zalomen.
\paragraph{Definice 1.}Zásobníkový automat (ZA) \emph{je definován jako
sedmice tvaru $A = (Q, \Sigma, \Gamma, \delta, q_0, Z_0, F)$, kde:}

\begin{itemize}
\item $Q$ je konečná množina vnitřních (řídicích) stavů,
\item $\Sigma$ je konečná vstupní abeceda,
\item $\Gamma$ je konečná zásobníková abeceda,
\item {$\delta$ je přechodová funkce $Q \times (\Sigma \cup {\epsilon}) \times \Gamma \rightarrow 2^{Q\times\Gamma^*},$}
\item $q_0 \in Q$ je počáteční stav, $Z_0 \in \Gamma$ je startovací symbol zásobníku a $F \subseteq Q$ je množina koncových stavů.
\end{itemize}

Nechť $P = (Q, \Sigma, \Gamma, \delta, q_0, Z_0, F)$ je zásobníkový automat. \emph{Konfigurací} nazveme trojici $(q, w, \alpha) \in Q\times\Sigma^{*}\times\Gamma^{*}$, kde $q$ je aktuální stav vnitřního řízení, $w$ je dosud nezpracovaná část vstupního řetězce a $\alpha = Z_{i1}Z_{i2}\dots Z_{ik}$ je obsah zásobníku\footnote{$Z_{i_{1}}$ je vrchol zásobníku}.

\subsection{Podsekce obsahující větu a odkaz}
\paragraph{Definice 2.}Řetězec $w$ nad abecedou $\Sigma$ je přijat ZA $A$ \emph{jestliže} $(q_0, w, Z_0)$ 
$\underset{A}{\overset{*}{\vdash}}$
$(q_F,\epsilon,\gamma)$ \emph{pro nějaké} $\gamma \in \Gamma^{*}$ \emph{a} $q_F \in F.$
\emph{Množinu} $L(A) =\{ w\  |\  w$ \emph{je přijat ZA A}\} $\subseteq \Sigma^*$ \emph{nazývame} jazyk přijímaný TS $M$.\\

Nyní si vyzkoušíme sazbu vět a důkazů opět s~použitím
balíku \texttt{amsthm}.

\paragraph{Věta 1.} \emph{Třída jazyků, které jsou přijímány ZA, odpovídá}
bezkontextovým jazykům.

\begin{proof} V~důkaze vyjdeme z~Definice 1 a 2.\end{proof} 

\section{Rovnice a odkazy}
Složitější matematické formulace sázíme mimo plynulý
text. Lze umístit několik výrazů na jeden řádek, ale pak je
třeba tyto vhodně oddělit, například příkazem \verb/\quad./
\bigskip

$\sqrt[i]{x_i^3}$ kde $x_i$ je $i$-té sudé číslo splňující $x_i^{2-x_i^{i^2}} \leq x_i^{y_i^3}$\\

V~rovnici (1) jsou využity tři typy závorek s~různou
explicitně definovanou velikostí.
\begin{equation}
x = \bigg[\Big\{\big[a+b\big]*c\Big\}^d\ominus 1\bigg]^{1/2}
\end{equation}

$$y = \lim_{x\to\infty}\frac{\frac{1}{log_{10}x}}{\sin^{2}x+\cos^{2}x} $$

V~této větě vidíme, jak vypadá implicitní vysázení limity $\lim_{n\to\infty} f(n)$ v~normálnom odstavci textu. Podobně je to i s~dalšími symboly jako
$\prod_{i=1}^n 2^i$ či $\bigcap_{A\in B} A$. V~případě vzorců $\lim\limits_{x\to\infty} f(n)$ a $\prod\limits_{i=1}^n 2^i$ jsme si vynutili méně úspornou sazbu příkazem \verb/\limits./
\begin{equation}
\int_b^a g(x)\,\mathrm{d}x = -\underset{a}{\overset{b}{\int}} f(x)\,\mathrm{d}x
\end{equation}

\begin{equation}
\overline{\overline{A \wedge B}} \Leftrightarrow \overline{\overline{A} \vee \overline{B}}
\end{equation}

\section{Matice}

Pro sázení matic se velmi často používá prostředí \texttt{array} a závorky (\verb/\left/,\verb/\right/).
\begin{equation*}
    \begin{bmatrix}
        &\widehat{\beta + \gamma} & \hat{\pi}\ \\
        \ \vec{a}& \overset{\longleftrightarrow}{AC} &
    \end{bmatrix} = 1 \Longleftrightarrow \mathbb{Q} = \mathbf{R}\\
\end{equation*}

$$\mathbf{A} = \begin{vmatrix}a_{11} & a_{12} & \dots & a_{1n}\\ a_{21} & a_{22} & \dots & a_{2n}\\ \vdots & \vdots & \ddots & \vdots\\ a_{m1} & a_{m2} & \dots & a_{mn}\end{vmatrix} = \begin{matrix}\ t & u \\ v & w\ \end{matrix} = tw - uv$$

Prostředí \texttt{array} lze úspěšně využít i jinde.

$$\begin{pmatrix}n\\k\end{pmatrix} = \left\{
\begin{array}{c l}
0 & \text{pro } k < 0 \  \text{nebo}\  k > n \\
\frac{n!}{k!(n-k)!} & \text{pro}\ 0 \leq k \leq n
\end{array} \right.$$







\end{document}