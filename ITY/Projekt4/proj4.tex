\documentclass[a4paper, 11pt]{article}
\usepackage[czech]{babel}
\usepackage[utf8]{inputenc}
\usepackage[left=2cm, top=3cm, text={17cm, 24cm}]{geometry}
\usepackage{times}
\usepackage[unicode]{hyperref}

\begin{document}

\begin{titlepage}
\begin{center}
\textsc{\Huge Fakulta informačních technologií\\
Vysoké učení technické v~Brně}\\
\vspace{\stretch{0.382}}
\LARGE Typografie a publikování -- 3. projekt\\
\Huge{Tabulky a~obrázky}\\
\vspace{\stretch{0.618}}
\Large \today \hfill         Tomáš Hrúz (xhruzt00)
\newpage
\end{center}
\end{titlepage}

\section{Typografie}

\subsection{Definice}
Typografie je věda o~tom, jak by měl vypadat tisknutý text. Není to ani zdaleka taká exaktní věda jako například matematika.
Z~veliké části je to umění. Cílem typografie je prezentovat text čtenářovi vhodným a~autorem zamýšleným spůsobem.
Bližší shrnutí nájdete v~dokumentu \cite{Forisek}.

\subsection{Písmo}
Kde by byla typografie kdyby nebylo písma, základ samotného textu. Účelem písma je čtení, tím pádem je potřeba lehká čitelnost. Něco málo k~témě najdete v~\cite{Kilianova}, případne podrobněj zhrnuto v~\cite{Cerny}.

\subsection{Fonty}
Existují dva typy fontů a~to je \verb/bitmap/ a \verb/outline/. První tvoří znaky pomocí pixelů, což znamená, že přiblížením se znak rozostřuje. Druhý typ se vykresluje matematicky pomocí ůseček a křivek. Bližší info najdete tady \cite{Felici}.

\subsection{Proč \LaTeX ?}
Otázka, která zajíma každého nováčka s~\LaTeX em . Proč pracovat s~něčím tak složitým, když můžem použít MS Word. 
V~\cite{Svanda} najdete odpověď na tuhle otázku. Netřeba se však nechat unést, protože \LaTeX\ má i~svá negativa viz \cite{Olsak}.

\subsection{Osobnosti}
 Za dobu co typografie existuje bylo i~mnoho skvělých typografů, o~kterých se dočtete zde \cite{Heller}

\subsection{Vizualní poezie}
V~19. století se někteří umělci rozhodli, že se nebudou řídit pravidly psání typicky rovně do řádku, a~to byl začátek vizualní poezie. Téma je dobře zhrnuta v~\cite{Hillner}.

\subsection{Program Asymptote}
Asymptote je open source nástroj pro tvorbu 2D a 3D grafiky. Jeho výhodou je společná práce s~\LaTeX em, protože grafiku můžeme generovat společne se samotným dokumentem. O~programu Asymptote se dočtete zde \cite{Kutal}.

\subsection{Koenig \& Bauer}
Koenig \& Bauer je nejstarší tisková manufaktura na světe. Oslavuje 200. narodeniny a~k~
tomuhle výročí plánuje změny v~imidžu firmy. Bližší info v~\cite{Neale}.





\newpage
\bibliographystyle{czechiso}
\bibliography{proj4}

\end{document}