\documentclass[a4paper,11pt]{article}
\usepackage[left=2cm,text={17cm, 24cm},top=3cm]{geometry}
\usepackage[IL2]{fontenc}
\usepackage[utf8]{inputenc}
\usepackage{graphics}
\usepackage{times}
\usepackage[czech]{babel}
\usepackage{algorithm2e}
\usepackage{multicol}
\usepackage{multirow}
\usepackage{hyperref}
\usepackage{graphicx}
\usepackage{pdflscape}


\begin{document}
\begin{titlepage}
\begin{center}
\textsc{\Huge Fakulta informačních technologií\\
Vysoké učení technické v~Brně}\\
\vspace{\stretch{0.382}}
\LARGE Typografie a publikování -- 3. projekt\\
\Huge{Tabulky a obrázky}\\
\vspace{\stretch{0.618}}
\Large \today \hfill         Tomáš Hrúz (xhruzt00)
\newpage
\end{center}
\end{titlepage}

\section{Úvodní strana}
Název práce umístěne do zlatého řezu~a nezapomeňte uvést dnešní datum a~vaše jméno a~příjmení.

\section{Tabulky}
Pro sázení tabulek můžeme použít buď prostředí \verb/tabbing/ nebo prostředí \verb/tabular/.

\subsection{Prostředí \texttt{tabbing}}
Při použití \verb/tabbing/ vypadá tabulka následovně:
\begin{tabbing}
\hspace*{3cm}\=\hspace*{1.5cm}\= \kill
\textbf{Ovoce} \> \textbf{Cena} \> \textbf{Množství} \\
Jablka\> 25,90 \> 3 kg \\
Hrušky\> 27,40 \> 2,5 kg \\
Vodní melouny \> 35,- \> 1 kus
\end{tabbing}
\bigskip


Toto prostředí se dá také použít pro sázení algoritmů, ovšem vhodnější je použít 
prostředí \verb/algorithm/ nebo \verb/algorithm2e/ (viz sekce 3).

\subsection{Prostředí tabular}
Další možností, jak vytvořit tabulku, je použít prostředí tabular. Tabulky pak budou vypadat takto\footnote{Kdyby byl problem s\texttt{ cline,} zkuste se podívat třeba sem:\href{http://www.abclinuxu.cz/tex/poradna/show/325037}{http://www.abclinuxu.cz/tex/poradna/show/325037}.} :
\bigskip
\begin{table}[ht]
\catcode`\-=12
\begin{center}
\begin{tabular}{| l | c | r |} \hline
& \multicolumn{2}{|c|}{\textbf{Cena}} \\ \cline{2-3}
\textbf{Měna} & \textbf{nákup} & \textbf{prodej} \\ \hline
EUR & 25,615 & 27,20\\
GBP & 29,899 & 31,80\\
USD & 22,571 & 25,51 \\ \hline
\end{tabular}
\caption{Tabulka kurzů k~dnešnímu dni}
\label{kurzy}
\end{center}
\end{table}

\begin{table}[h]
\catcode`\-=12
\begin{center}		
\begin{tabular}[p]{| c | c |}\hline
$ A $	 & $ {\neg}A $	\\ \hline
\textbf{P} & N	\\ \hline
\textbf{O} & O~\\ \hline
\textbf{X} & X	\\ \hline
\textbf{N} & P	\\ \hline
\end{tabular}
\begin{tabular}[p]{| c | c | c | c | c | c |}
\hline
\multicolumn{2}{| c |}{\multirow{2}{*}{$ A \wedge B $}} & \multicolumn{4}{c |}{$ B $}\\ \cline{3-6}
\multicolumn{2}{| c |}{} & \textbf{P} & \textbf{O} & \textbf{X}	& \textbf{N} \\ \hline
\multirow{4}{*}{$ A $} & \textbf{P} & P & O~& X & N \\ \cline{2-6}
& \textbf{O} & O~& O~& N & N \\ \cline{2-6}
& \textbf{X} & X & N & X & N \\ \cline{2-6}
& \textbf{N} & N & N & N & N \\ \hline
\end{tabular}
\begin{tabular}[p]{| c | c | c | c | c | c |}
\hline
\multicolumn{2}{| c |}{\multirow{2}{*}{$ A \vee B $}} & \multicolumn{4}{c |}{$ B $}\\ \cline{3-6}
\multicolumn{2}{| c |}{} & \textbf{P} & \textbf{O} & \textbf{X}	& \textbf{N} \\ \hline
\multirow{4}{*}{$ A $} & \textbf{P} & P & P & P & P \\ \cline{2-6}
& \textbf{O} & P & O~& P & O~\\ \cline{2-6}
& \textbf{X} & P & P & X & X \\ \cline{2-6}
& \textbf{N} & P & O~& X & N \\ \hline
\end{tabular}
\begin{tabular}[p]{| c | c | c | c | c | c |}
\hline
\multicolumn{2}{| c |}{\multirow{2}{*}{$ A \rightarrow B $}} & \multicolumn{4}{c |}{$ B $}\\ \cline{3-6}
\multicolumn{2}{| c |}{} & \textbf{P} & \textbf{O} & \textbf{X}	& \textbf{N} \\ \hline
\multirow{4}{*}{$ A $} & \textbf{P} & P & O~& X & N \\ \cline{2-6}
& \textbf{O} & P & O~& P & O~\\ \cline{2-6}
& \textbf{X} & P & P & X & X \\ \cline{2-6}
& \textbf{N} & P & P & P & P \\ \hline
\end{tabular}
\caption{Protože Kleeneho trojhodnotová logika už je \uv{zastaralá}, uvádíme si zde příklad čtyřhodnotové logiky}
\label{logika}
	\end{center}
	\end{table}

\pagebreak

\section{Algoritmy}
\label{algoritmus}
Pokud budeme chtít vysázet algoritmus, můžeme použít prostředí \verb/algorithm/\footnote{Pro nápovědu, jak zacházet s~prostředím algorithm, můžeme zkusit tuhle stránku:\\ \href{http://ftp.cstug.cz/pub/tex/CTAN/macros/latex/contrib/algorithms/algorithms.pdf}{http://ftp.cstug.cz/pub/tex/CTAN/macros/latex/contrib/algorithms/algorithms.pdf}.} nebo \verb/algorithm2e/\footnote{Pro \texttt{algorithm2e} zase tuhle:
\href{http://ftp.cstug.cz/pub/tex/CTAN/macros/latex/contrib/algorithm2e/algorithm2e.pdf}{http://ftp.cstug.cz/pub/tex/CTAN/macros/latex/contrib/algorithm2e/algorithm2e.pdf}.}.
Příklad použití prostředí \verb/algorithm2e/ viz Algoritmus 1.

\begin{algorithm}[H]
\caption{\textsc{FastSLAM}}
\KwIn{$ (X_{t - 1}, u_t, z_t) $}
\KwOut{$ X_t $} 
$ \overline{X_t} = X_t = 0 $ \\
\For{$ k = 1 \textrm{\emph{ to }} M $}
{
$ x_t^{[k]} = \emph{sample\_motion\_model}(u_t, x_{t - 1}^{[k]}) $ \\
$ \omega_t^{[k]} = \emph{measurement\_model}(z_t, x_t^{[k]}, m_{t - 1}) $ \\
$ m_t^{[k]} = updated\_occupancy\_grid(z_t, x_t^{[k]}, m_{t - 1}^{[k]}) $ \\
$ \overline{X_t} = \overline{X_t} + \langle x_x^{[m]}, \omega_t^{[m]}  \rangle $ \\
}
\For{$ k = 1 \textrm{\emph{ to }} M $}
{
draw $ i $ with probability $ \approx\omega_t^{[i]} $ \\
add $ \langle x_x^{[k]}, m_t^{[k]} \rangle\textrm{ to } X_t $ \\
}
\Return{$ X_t $}
\end{algorithm}

\section{Obrázky}
Do našich článků můžeme samozřejmě vkládat obrázky. Pokud je obrázkem fotografie,
můžeme klidně použít bitmapový soubor. Pokud by to ale mělo být nějaké schéma nebo
něco podobného, je dobrým zvykem takovýto obrázek vytvořit vektorově.

\begin{figure}[h]
\begin{center}
\scalebox{0.4}{\includegraphics{etiopan.eps}\reflectbox{\includegraphics{etiopan.eps}}}
\caption{Malý Etiopánek a~jeho bratříček}
\label{etiopan}
\end{center}
\end{figure}


\newpage
Rozdíl medzi vektorovým\dots

\begin{figure}[h]
\begin{center}
\scalebox{0.4}{\includegraphics{oniisan.eps}}
\caption{Vektorový obrázek}
\label{vektor}
\end{center}
\end{figure}

\dots a~bitmapovým obrázkem

\begin{figure}[h]
\begin{center}
\scalebox{0.6}{\includegraphics{oniisan2.eps}}
\caption{Bitmapový obrázek}
\label{bitmap}
\end{center}
\end{figure}

se projeví například pri zvětšení.\\
Odkazy (nejen ty) na obrázky~\ref{etiopan},~\ref{vektor} a~\ref{bitmap}, na  
tabulky~\ref{kurzy} a~\ref{logika} a také na algoritmus~\ref{algoritmus} jsou udělány pomocí 
křížových odkazů. Pak je ovšem potřeba zdrojový soubor přeložit dvakrát.
\quad

Vektorové obrázky lze vytvořit i~přímo v~\LaTeX u, například pomocí prostředí 
\verb/picture/.

\newpage
\begin{landscape}
\begin{figure}[h]
\setlength{\unitlength}{1mm}
\centering
\begin{picture}(200, 110)
\linethickness{1pt}
\put(0, 0){\framebox(200, 100){}} % rám

\put(30, 70){\circle{20}} %slnko

\linethickness{1.5mm}
\put(5,5){\line(1,0){190}} %zem

\linethickness{0.6mm}
\put(20,5){\line(0,0){30}}
\put(20,35){\line(1,0){100}}
\put(120,35){\line(0,-1){30}}
\put(120,5){\line(0,0){75}}
\put(120,80){\line(1,0){20}}
\put(140,80){\line(0,-1){75}}
\put(140,80){\line(0,0){10}}
\put(140,90){\line(1,0){30}}
\put(170,90){\line(0,-1){85}} %obrysy budovy

\linethickness{0.5mm}
\put(40,5){\line(0,0){20}}
\put(70,5){\line(0,0){20}}
\linethickness{0.8mm}
\put(38,25){\line(1,0){34}} %vchod

\linethickness{0.3mm}
\put(41,5){\line(0,0){3}}
\put(69,5){\line(0,0){3}}
\put(41,8){\line(1,0){28}}
\put(43,8){\line(0,0){3}}
\put(67,8){\line(0,0){3}}
\put(43,11){\line(1,0){24}} %schody

\put(150,10){\line(0,0){80}}
\put(160,10){\line(0,0){80}}
\put(150,10){\line(1,0){10}}
\put(150,20){\line(1,0){10}}
\put(150,30){\line(1,0){10}}
\put(150,40){\line(1,0){10}}
\put(150,50){\line(1,0){10}}
\put(150,60){\line(1,0){10}}
\put(150,70){\line(1,0){10}}
\put(150,80){\line(1,0){10}}

\linethickness{0.1mm}
\put(150,14){\line(1,0){10}}
\put(150,24){\line(1,0){10}}
\put(150,34){\line(1,0){10}}
\put(150,44){\line(1,0){10}}
\put(150,54){\line(1,0){10}}
\put(150,64){\line(1,0){10}}
\put(150,74){\line(1,0){10}}
\put(150,84){\line(1,0){10}}

\put(151,10){\line(0,0){4}}
\put(152,10){\line(0,0){4}}
\put(153,10){\line(0,0){4}}
\put(154,10){\line(0,0){4}}
\put(155,10){\line(0,0){4}}
\put(156,10){\line(0,0){4}}
\put(157,10){\line(0,0){4}}
\put(158,10){\line(0,0){4}}
\put(159,10){\line(0,0){4}}

\put(151,20){\line(0,0){4}}
\put(152,20){\line(0,0){4}}
\put(153,20){\line(0,0){4}}
\put(154,20){\line(0,0){4}}
\put(155,20){\line(0,0){4}}
\put(156,20){\line(0,0){4}}
\put(157,20){\line(0,0){4}}
\put(158,20){\line(0,0){4}}
\put(159,20){\line(0,0){4}}

\put(151,30){\line(0,0){4}}
\put(152,30){\line(0,0){4}}
\put(153,30){\line(0,0){4}}
\put(154,30){\line(0,0){4}}
\put(155,30){\line(0,0){4}}
\put(156,30){\line(0,0){4}}
\put(157,30){\line(0,0){4}}
\put(158,30){\line(0,0){4}}
\put(159,30){\line(0,0){4}}

\put(151,40){\line(0,0){4}}
\put(152,40){\line(0,0){4}}
\put(153,40){\line(0,0){4}}
\put(154,40){\line(0,0){4}}
\put(155,40){\line(0,0){4}}
\put(156,40){\line(0,0){4}}
\put(157,40){\line(0,0){4}}
\put(158,40){\line(0,0){4}}
\put(159,40){\line(0,0){4}}

\put(151,50){\line(0,0){4}}
\put(152,50){\line(0,0){4}}
\put(153,50){\line(0,0){4}}
\put(154,50){\line(0,0){4}}
\put(155,50){\line(0,0){4}}
\put(156,50){\line(0,0){4}}
\put(157,50){\line(0,0){4}}
\put(158,50){\line(0,0){4}}
\put(159,50){\line(0,0){4}}

\put(151,60){\line(0,0){4}}
\put(152,60){\line(0,0){4}}
\put(153,60){\line(0,0){4}}
\put(154,60){\line(0,0){4}}
\put(155,60){\line(0,0){4}}
\put(156,60){\line(0,0){4}}
\put(157,60){\line(0,0){4}}
\put(158,60){\line(0,0){4}}
\put(159,60){\line(0,0){4}}

\put(151,70){\line(0,0){4}}
\put(152,70){\line(0,0){4}}
\put(153,70){\line(0,0){4}}
\put(154,70){\line(0,0){4}}
\put(155,70){\line(0,0){4}}
\put(156,70){\line(0,0){4}}
\put(157,70){\line(0,0){4}}
\put(158,70){\line(0,0){4}}
\put(159,70){\line(0,0){4}}

\put(151,80){\line(0,0){4}}
\put(152,80){\line(0,0){4}}
\put(153,80){\line(0,0){4}}
\put(154,80){\line(0,0){4}}
\put(155,80){\line(0,0){4}}
\put(156,80){\line(0,0){4}}
\put(157,80){\line(0,0){4}}
\put(158,80){\line(0,0){4}}
\put(159,80){\line(0,0){4}} %balkony






\end{picture}
\caption{Vektorový obrázek moderních kolejí.}
\end{figure}
\end{landscape}
\end{document}